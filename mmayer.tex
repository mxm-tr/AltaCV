%%%%%%%%%%%%%%%%%
% This is an example CV created using altacv.cls (v1.1.5, 1 December 2018) written by
% LianTze Lim (liantze@gmail.com), based on the
% Cv created by BusinessInsider at http://www.businessinsider.my/a-sample-resume-for-marissa-mayer-2016-7/?r=US&IR=T
%
%% It may be distributed and/or modified under the
%% conditions of the LaTeX Project Public License, either version 1.3
%% of this license or (at your option) any later version.
%% The latest version of this license is in
%%    http://www.latex-project.org/lppl.txt
%% and version 1.3 or later is part of all distributions of LaTeX
%% version 2003/12/01 or later.
%%%%%%%%%%%%%%%%

%% If you are using \orcid or academicons
%% icons, make sure you have the academicons
%% option here, and compile with XeLaTeX
%% or LuaLaTeX.
% \documentclass[10pt,a4paper,academicons]{altacv}

%% Use the "normalphoto" option if you want a normal photo instead of cropped to a circle
%\documentclass[10pt,a4paper,normalphoto]{altacv}

\documentclass[10pt,a4paper,ragged2e]{altacv}

%% AltaCV uses the fontawesome and academicon fonts
%% and packages.
%% See texdoc.net/pkg/fontawecome and http://texdoc.net/pkg/academicons for full list of symbols. You MUST compile with XeLaTeX or LuaLaTeX if you want to use academicons.

% Change the page layout if you need to
\geometry{left=0.8cm,right=7.8cm,marginparwidth=6.2cm,marginparsep=0.8cm,top=1cm,bottom=0cm}

% Change the font if you want to, depending on whether
% you're using pdflatex or xelatex/lualatex
\ifxetexorluatex
  % If using xelatex or lualatex:
  \setmainfont{Lato}
\else
  % If using pdflatex:
  \usepackage[utf8]{inputenc}
  \usepackage[T1]{fontenc}
  \usepackage[default]{lato}
\fi

% Change the colours if you want to
\definecolor{VividPurple}{HTML}{3E0097}
\definecolor{SlateGrey}{HTML}{2E2E2E}
\definecolor{LightGrey}{HTML}{666666}
\colorlet{heading}{VividPurple}
\colorlet{accent}{VividPurple}
\colorlet{emphasis}{SlateGrey}
\colorlet{body}{LightGrey}

% Change the bullets for itemize and rating marker
% for \cvskill if you want to
\renewcommand{\itemmarker}{{\small\textbullet}}
\renewcommand{\ratingmarker}{\faCircle}

%% sample.bib contains your publications
\addbibresource{sample.bib}

\begin{document}
\name{Maxime\hspace{0.56em}Hubert}
\tagline{Computer Science Engineer in the Cloud}
% Cropped to square from https://en.wikipedia.org/wiki/Marissa_Mayer#/media/File:Marissa_Mayer_May_2014_(cropped).jpg, CC-BY 2.0
%\photo{2.5cm}{mmayer-wikipedia-cc-by-2_0}
\personalinfo{%
  % Not all of these are required!
  % You can add your own with \printinfo{symbol}{detail}
  \email{maxime.hubert@outlook.com}
%   \phone{000-00-0000}
  \mailaddress{7566 8th street NW}
  \location{Washington, DC}
%  \homepage{marissamayr.tumblr.com/}
%  \twitter{@marissamayer}
  \linkedin{www.linkedin.com/in/maximehubert}
  \age{2+ years exp.}
%   \github{github.com/mmayer} % I'm just making this up though.
%   \orcid{orcid.org/0000-0000-0000-0000} % Obviously making this up too. If you want to use this field (and also other academicons symbols), add "academicons" option to \documentclass{altacv}
}

%% Make the header extend all the way to the right, if you want.
\begin{fullwidth}
\makecvheader
\end{fullwidth}

%% Depending on your tastes, you may want to make fonts of itemize environments slightly smaller
\AtBeginEnvironment{itemize}{\small}

%% Provide the file name containing the sidebar contents as an optional parameter to \cvsection.
%% You can always just use \marginpar{...} if you do
%% not need to align the top of the contents to any
%% \cvsection title in the "main" bar.
\cvsection[sidebar_page1]{Experience}


\cvevent{Research Engineer (Since April 2017)}{National Institute of Standards and Technology (NIST)}{}{Washington DC area, USA}

\cvtag{IARPA}\cvtag{DIVA}\cvtag{Video processing} \hfill
\cvcalendar{2018 -- Now }\hspace{5em}\vspace{0.3em}\vspace{0.3em}\break
Part of a team that designed an evaluation process for Activity Detection algorithms on video footage for the IARPA Deep Intermodal Video Analytics program.
\begin{itemize}

\item Allowed the Continuous Delivery and validation of video processing systems using IaaS/PaaS technologies on a GPU-accelerated cluster. Performed evaluation of the performance of video processing algorithms in an automated manner using jobs scheduling and resources abstraction.
\item 10+ GPU-ready systems over 100+ hours of video sets were evaluated and monitored.
\item Speaker in a talk about this work at the Open Infrastructure Summit 2019 in Denver, CO.
\end{itemize}
\textbf{\color{accent}Tags:} Openstack, GPU cluster, Job scheduling, Data Science, CD, Infrastructure as Code.

\divider

\cvtag{DARPA} \cvtag{D3M} \cvtag{MLaaS} \hfill \cvcalendar{2017 -- 2018} \hspace{5em}\vspace{0.3em}\vspace{0.3em}\break
Part of a team that developed, integrated, and supported an evaluation framework for Machine Learning algorithms on a cluster for the DARPA Data Driven Discovery of Models program.
\begin{itemize}
\item Allowed a Continuous Delivery and validation of ML systems, as well as the automation of performance evaluation using jobs definitions and scheduling mechanisms.
\item 20+ ML systems over 70+ data sets were evaluated and monitored. This work included the deployment of web applications for human-assisted ML.
\end{itemize}
\textbf{\color{accent}Tags:} Kubernetes, ELK, Openstack, Docker, Python, Machine Learning, Job scheduling, Data science, CD.

\divider

\cvtag{IARPA}\cvtag{MATERIAL}\cvtag{Machine Translation} \hfill
\cvcalendar{2017 -- Now }\hspace{5em}\vspace{0.3em}\vspace{0.3em}\break
Maintained data sets and code for the IARPA Machine Translation for English Retrieval of Information in Any Language (MATERIAL) program.

\begin{itemize}
\item Implemented metrics calculators, data management scripts, and interactive analyzers to read the performance of MT information retrieval systems.
\item Worked towards standardizing, version-controlling, and documenting the different steps of the evaluation pipeline: from the MT data sets collection to the calculation of MT systems performance metrics.
\end{itemize}
\textbf{\color{accent}Tags:} Data analysis, Data standardization, Python, Numpy, Pandas, Jupyter.

\divider

\cvevent{ Internship: Network Support Technician (Summer 2016)}{ Crown Holdings }{}{Wantage Oxfordshire, UK}
\cvtag{Network support}\cvtag{WAN}\vspace{0.3em}\break
\begin{small}
Part of a team of network technicians that provided support, troubleshooting and maintenance on an international company network.
\begin{itemize}
 \item Provided users with remote assistance.
 
 \item Worked on updating the process to connect industrial monitors to the network by redacting configuration procedures.
\end{itemize}
\textbf{\color{accent}Tags:} WAN, Network, Support.
\end{small}

\divider

\textbf{\large{University projects:}}

\smallskip
\small\textbf{Ministry of Defense, France}\hfill\cvcalendar{2017 -- 6 months}\hspace{5em}

\small{Part of a team that prototyped an embedded solution to interact with flash memory.}
\newline \smallskip
\textbf{L'Oréal \& CRAN, France}\hfill\cvcalendar{2016 -- 2 months}\hspace{5em}

\small{Studied the feasibility of an AI algorithm performing image classification.}

\clearpage

%\cvsection[page2sidebar]{Publications}

%\nocite{*}

%\printbibliography[heading=pubtype,title={\printinfo{\faBook}{Books}},type=book]

%\divider

%\printbibliography[heading=pubtype,title={\printinfo{\faFileTextO}{Journal Articles}}, type=article]

%\divider

%\printbibliography[heading=pubtype,title={\printinfo{\faGroup}{Conference Proceedings}},type=inproceedings]

%% If the NEXT page doesn't start with a \cvsection but you'd
%% still like to add a sidebar, then use this command on THIS
%% page to add it. The optional argument lets you pull up the
%% sidebar a bit so that it looks aligned with the top of the
%% main column.
% \addnextpagesidebar[-1ex]{page3sidebar}


\end{document}
